% LaTeX resume using res.cls
\documentclass[resmargin]{res}
\usepackage[top = .7in, bottom = .5in, left = 1in]{geometry}
\usepackage{enumitem}
%\usepackage{helvetica} % uses helvetica postscript font (download helvetica.sty)
%\usepackage{newcent}   % uses new century schoolbook postscript font 
\setlength{\textwidth}{5.1in} % set width of text portion
%\setlist{nolistsep}

\begin{document}

% Center the name over the entire width of resume:
\moveleft.5\hoffset\vbox{\centerline{\Huge\bf Jesse Zhang}}
% Draw a horizontal line the whole width of resume:
 \moveleft\hoffset\vbox{\hrule width\resumewidth height 1pt}\smallskip
% address begins here
% Again, the address lines must be centered over entire width of resume:
\address{\small 404 E. Stoughton St\\
\small Apt \#4\\
\small Champaign, IL 61820}
\address{\small xzhan121@illinois.edu\\
\small cell: 502.510.4947\\
\small github: macisasandwich}


\begin{resume}

\vspace{-3mm}
\section{EDUCATION} {\sl Bachelor of Science,} Computer Engineering \\
                      % \sl will be bold italic in New Century Schoolbook (or
	              % any postscript font) and just slanted in
		      %	Computer Modern (default) font
                University of Illinois, Urbana, Illinois \\
                Relevant Coursework: ECE 391 Computer Systems Engineering, \\CS 225 Data Structures, ECE 290 Computer Engineering I,\\ECE 210 Analog Signal Processing, INFO 490 Data Science
                
\vspace{-4mm}
\section{AWARDS}
\textnormal {National Science Foundation REU Fellowship} 

\vspace{-3mm}
\section{PUBLICATIONS}
\textnormal{Kim, M., \textbf{Zhang, X.}, Milenkovic, O. (2014). \textit{Parallel Compression of Metagenomic Sequences via Extended Golomb Codes} Selected for a platform presentation at the Biological Data Science Workshop, Cold Spring Harbor Laboratory, November 2014}

\vspace{-3mm}
\section{WORK \\ EXPERIENCE} 
{\bf Fulcrum GT} -- {\sl Software Engineering Intern}\hfill Summer 2015
	\begin{itemize} \itemsep -2pt
	\item Primary iOS backend developer -- responsible for designing and implementing the data model in Core Data for a legal time entry solution
	\item Explored location and physical activity tracking, as well as geo-fencing, using Core Location and Core Motion frameworks
	\item Designed overall program flow for asynchronous activities using NSNotificationCenter, GCD, delegates, and closures
	\end{itemize}
\vspace{-4mm}
{\bf Coordinated Science Lab} -- {\sl Research Intern }\hfill Summer 2014 - Summer 2015
                 \begin{itemize}  \itemsep -2pt %reduce space between items
                 \item Intern with the Bioinformatics Group of the ECE Department at UIUC 
                 \item Worked with Perl and Java to automate parallelized DNA 
                 {\mbox compression} and maximize DNA compression ratio
                 \item Developed the Extended Golomb Code compression scheme adapted for DNA read-specific statistical distributions
                \end{itemize}
 
\vspace{-4mm}
\section{PROJECTS} 
{\bf ECE 391} -- {\sl x86 Assembly, C} \\ 
\textnormal {Za Big New OS -- Linux-like operating system} 
\begin{itemize} \itemsep -2pt
\vspace {.5mm}
\item Implemented the PIC configuration code and developed the interrupt handlers for the keyboard and RTC
\item Implemented the Linux ext2 file system with both read and write functionality
\item Developed the system calls for device and file I/O as well as the execution and halting of a task
\item Implemented the C Standard Library as well as C runtime in conjunction with the native runtime 
\end{itemize}
\vspace{-3mm}
{\bf BoilerMake} -- {\sl C, Java, Objective-C} \\
\textnormal {HackedReality -- virtual reality using Google Cardboard} (\sl {Winning project 2014})
\begin{itemize} \itemsep -2pt
\item Developed a driver for a DDR Dancepad to mimic the omni-directional treadmill and implemented dynamic remapping of the dancepad buttons
\item Used the magnetometer in Android phone to track the user's orientation
\item Used the Pebble smartwatch to track the user's body motions
\end{itemize}
\vspace{-3mm}
{\bf EXT9000} -- Linux EXT2/3/4 interactive parser in C for fun \hfill {\sl Work in progress} 

\vspace{-4mm}
\section {ACTIVITIES}

{\bf IEEE@UIUC}
\begin{itemize} \itemsep -2pt 
 \item IEEE Projects Committee
 \item IEEE Hackathon 2013 -- {\sl Java}
 \begin{itemize}
  \setlength{\itemsep}{2pt}
  \vspace{-2mm}
  \item PairTunes - a music streaming application that creates ad-hoc server-client networks across multiple computers to achieve concurrent surround sound playback
 \end{itemize}
\end{itemize}

\vspace{-5mm}
{\bf ACM@UIUC}
\begin{itemize} \itemsep -2pt
 \item SIGDave - various short-term projects
 \item SigOPS - Special interest group in operating systems
\end{itemize} 

\vspace{-3mm}
\section{TECHNICAL \\ SKILLS} {\sl Languages:} C, x86 Assembly, Swift, C++, Perl, Java, Objective-C

\end{resume}
\end{document}




